\documentclass[11pt]{article}
\usepackage{euscript}

\usepackage{amsmath}
\usepackage{amsthm}
\usepackage{amssymb}
\usepackage{epsfig}
\usepackage{xspace}
\usepackage{color}
\usepackage{url}
\usepackage{enumerate}
\usepackage{subfigure}

%%%%%%%  For drawing trees  %%%%%%%%%
\usepackage{tikz}
\usetikzlibrary{calc, shapes, backgrounds}

%%%%%%%%%%%%%%%%%%%%%%%%%%%%%%%%%
\setlength{\textheight}{9in}
\setlength{\topmargin}{-0.600in}
\setlength{\headheight}{0.2in}
\setlength{\headsep}{0.250in}
\setlength{\footskip}{0.5in}
\flushbottom
\setlength{\textwidth}{6.5in}
\setlength{\oddsidemargin}{0in}
\setlength{\evensidemargin}{0in}
\setlength{\columnsep}{2pc}
\setlength{\parindent}{0em}
%%%%%%%%%%%%%%%%%%%%%%%%%%%%%%%%%


\newcommand{\eps}{\varepsilon}

\renewcommand{\c}[1]{\ensuremath{\EuScript{#1}}}
\renewcommand{\b}[1]{\ensuremath{\mathbb{#1}}}
\newcommand{\s}[1]{\textsf{#1}}

\newcommand{\E}{\textbf{\textsf{E}}}
\renewcommand{\Pr}{\textbf{\textsf{Pr}}}

\newenvironment{pkl}{%
\begin{itemize}%
%\vspace{-\topsep}%
\setlength\itemsep{-0.5\parskip}%
\setlength\parsep{0in}%
}{%
%\vspace{-\topsep}%
\end{itemize}}

\newenvironment{enu}{%
\begin{enumerate}%
\vspace{-\topsep}%
\setlength\itemsep{-\parskip}%
\setlength\parsep{0in}%
}{%
\vspace{-\topsep}%
\end{enumerate}}
\allowdisplaybreaks


\title{Readme for reproducibility submission of SIGMOD'21 paper ID 68}
\author{}
\date{}

\begin{document}
\maketitle

{\bf Source code info}

Repository: https://github.com/Yanqing-UTAH/ATTPCode\\
Programming Langauge: mainly C/C++, some of the scripts are written in
Python 2\\
Required software: gcc/g++ >= 8 (requires support for -std=gnu++17),
lapack and lapacke, blas and cblas, fftw3, python3 and the standard
shell programs (bash, grep, sed, and etc.).\\
Optional software: jupyter notebook, python2 (for plotting
figures); m4, autoconf, autoheader (not needed unless the ./configure
script does not work).


{\bf Test environment}
Software environment. Note that these are the ones installed at the
time we performed the experiments but it should be ok if you use later
versions of these packages.  If you're using Ubuntu, you should be
able to use ``apt install'' to get all these packages.
\begin{pkl}
	\item Ubuntu 18.04.6 LTS
	\item GCC/G++ 8.3.0
	\item liblapack-dev 3.7.1
	\item liblapacke-dev 3.7.1
	\item libblas-dev 3.7.1
	\item libatlas-base-dev 3.10.3
	\item libfftw3-dev 3.3.7
	\item (optional) m4 1.4.18
	\item (optional) autoconf 2.69
	\item (optional) autoheader 2.69
	\item (optional) for plotting figures: python2 and jupyter
		notebook
\end{pkl}

Hardware environment: The experiments are single-threaded and we only
launch one experiement at a time on a single node so the number of
cores or hyperthreading is irrelevant to our purpose. Cache size is
also irrelevant for our purpose as our data structure size far exceeds
even the L3 cache size (listed below just for reference). On the other
hand, we have 10 type-1 nodes and 6 type-2 nodes and we may launch any
of the experiments on any one of them depending on the availability.
We made sure that there were no computation or I/O heavy programs
running concurrently.
\begin{tabular}{|l|l|l|}
	\hline
	Node type &  1 & 2 \\\hline
	CPU & Intel Core i7-3820 & Intel Xeon E5-1650 v3 \\\hline
	CPU Frequency & 3.6 GHz & 3.50 GHz \\\hline
	L1 Cache & 32KB + 32 KB & 32KB + 32 KB\\\hline
	L2 Cache & 256 KB & 256 KB\\\hline
	L3 Cache & 10 MB (shared) & 15 MB (shared) \\\hline
	Memory & 64GB (DDR3-1600 x 8) & 128GB (DDR4-2133 x 4)
	Secondary storage & WD HDD 7200RPM 2TB & Seagate HDD 7200RPM 1TB
	\\\hline
	Network & not used & not used \\\hline
\end{tabular}



\end{document}

