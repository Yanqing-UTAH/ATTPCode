\documentclass[11pt]{article}
\usepackage{euscript}

\usepackage{amsmath}
\usepackage{amsthm}
\usepackage{amssymb}
\usepackage{epsfig}
\usepackage{xspace}
\usepackage{color}
\usepackage{url}
\usepackage{enumerate}
\usepackage{subfigure}

%%%%%%%  For drawing trees  %%%%%%%%%
\usepackage{tikz}
\usetikzlibrary{calc, shapes, backgrounds}

%%%%%%%%%%%%%%%%%%%%%%%%%%%%%%%%%
\setlength{\textheight}{9in}
\setlength{\topmargin}{-0.600in}
\setlength{\headheight}{0.2in}
\setlength{\headsep}{0.250in}
\setlength{\footskip}{0.5in}
\flushbottom
\setlength{\textwidth}{6.5in}
\setlength{\oddsidemargin}{0in}
\setlength{\evensidemargin}{0in}
\setlength{\columnsep}{2pc}
\setlength{\parindent}{0em}
%%%%%%%%%%%%%%%%%%%%%%%%%%%%%%%%%


\newcommand{\eps}{\varepsilon}

\renewcommand{\c}[1]{\ensuremath{\EuScript{#1}}}
\renewcommand{\b}[1]{\ensuremath{\mathbb{#1}}}
\newcommand{\s}[1]{\textsf{#1}}

\newcommand{\E}{\textbf{\textsf{E}}}
\renewcommand{\Pr}{\textbf{\textsf{Pr}}}

\newenvironment{pkl}{%
\begin{itemize}%
%\vspace{-\topsep}%
\setlength\itemsep{-0.5\parskip}%
\setlength\parsep{0in}%
}{%
%\vspace{-\topsep}%
\end{itemize}}

\newenvironment{enu}{%
\begin{enumerate}%
\vspace{-\topsep}%
\setlength\itemsep{-\parskip}%
\setlength\parsep{0in}%
}{%
\vspace{-\topsep}%
\end{enumerate}}
\allowdisplaybreaks


\title{Readme for reproducibility submission of SIGMOD'21 paper ID 68}
\author{}
\date{}

\begin{document}
\maketitle

\section{Source code info}

{\bf Repository:} https://github.com/Yanqing-UTAH/ATTPCode\\
{\bf Programming Langauge:} mainly C/C++, some of the scripts are written in
Python\\
{\bf Required software/library:} gcc/g++ >= 8 (requires support for
-std=gnu++17), make, lapack and lapacke, blas and cblas, fftw3,
python3, numpy, scipy, sklearn and the common shell programs (bash,
grep, sed, and etc.).\\
{\bf Optional software/library:} for plotting figures: jupyter notebook, python2
matplotlib, pandas, numpy; for recreating ``configure'' script if that
does not work in your environment: m4, autoconf, autoheader.  \\
{\bf Hardware requirement:} we assume the architecture implements <= 48-bit
virtual address space and always uses x86-64 canonical addresses.
That's the case with any Intel/AMD processor other than Intel Ice Lake
(which supports 5-level paging) as of right now (2021).

\section{Test environment}
{\bf Software environment:}
\begin{pkl}
	\item Ubuntu 18.04.6 LTS
    \item GNU make 4.1
	\item GCC/G++ 8.3.0
	\item liblapack-dev 3.7.1
	\item liblapacke-dev 3.7.1
	\item libblas-dev 3.7.1
	\item libatlas-base-dev 3.10.3
	\item libfftw3-dev 3.3.7
    \item python3 3.6.9, sklearn, scipy, numpy
    \item sklearn, scipy and numpy (maybe python3-pip
\end{pkl}
Note that these are the ones installed at the
time we performed the experiments but it should be ok if you use newer
versions. If you're using Ubuntu, you should be able to use the
following to get all these packages:
\begin{verbatim}
sudo apt install gcc g++ make liblapacke-dev libatlas-base-dev libfftw3-dev
pip install sklearn scipy numpy
\end{verbatim}

\newpage
{\bf Hardware environment:}

\vspace{1mm}
\begin{tabular}{|l|l|l|}
	\hline
	Node type &  1 & 2 \\\hline
	CPU & Intel Core i7-3820 & Intel Xeon E5-1650 v3 \\\hline
	CPU Frequency & 3.6 GHz & 3.50 GHz \\\hline
	L1 Cache & 32KB + 32 KB & 32KB + 32 KB\\\hline
	L2 Cache & 256 KB & 256 KB\\\hline
	L3 Cache & 10 MB (shared) & 15 MB (shared) \\\hline
	Memory & 64GB (DDR3-1600 x 8) & 128GB (DDR4-2133 x 4) \\\hline
	Secondary storage & WD HDD 7200RPM 2TB & Seagate HDD 7200RPM 1TB
	\\\hline
	Network & not used & not used \\\hline
\end{tabular}
\vspace{1mm}

The experiments are single-threaded and we only
launch one experiement at a time on a single node so the number of
cores or hyperthreading is irrelevant to our purpose. Cache size is
also irrelevant for our purpose as our data structure size far exceeds
even the L3 cache size (listed below just for reference). On the other
hand, we have 10 type-1 nodes and 6 type-2 nodes and we may launch any
of the experiments on any one of them depending on the availability.
We made sure that there were no computation or I/O heavy programs
running concurrently.

\section{Preparing datasets}

We created datasets based on the FIFA World Cup 98 website access logs
for the ATTP/BITP heavy hitter experiments and generated synthetic
datasets for the ATTP/BITP frequent direction experiments. Note these
scripts assumes python3 is available as ``python3'' and a GNU c++
compiler as ``g++'' in your environment.

For the world-cup datasets, run:
\begin{verbatim}
./data_proc/world-cup/prepare_data.sh
\end{verbatim}



The
following list summarizes all the datasets:

\begin{tabular}{}
\end{tabular}


\end{document}

