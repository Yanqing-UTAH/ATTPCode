\documentclass[11pt]{article}
\usepackage{euscript}

\usepackage{amsmath}
\usepackage{amsthm}
\usepackage{amssymb}
\usepackage{epsfig}
\usepackage{xspace}
\usepackage{color}
\usepackage{url}
\usepackage{enumerate}
\usepackage{subfigure}

%%%%%%%  For drawing trees  %%%%%%%%%
\usepackage{tikz}
\usetikzlibrary{calc, shapes, backgrounds}

%%%%%%%%%%%%%%%%%%%%%%%%%%%%%%%%%
\setlength{\textheight}{9in}
\setlength{\topmargin}{-0.600in}
\setlength{\headheight}{0.2in}
\setlength{\headsep}{0.250in}
\setlength{\footskip}{0.5in}
\flushbottom
\setlength{\textwidth}{6.5in}
\setlength{\oddsidemargin}{0in}
\setlength{\evensidemargin}{0in}
\setlength{\columnsep}{2pc}
\setlength{\parindent}{0em}
%%%%%%%%%%%%%%%%%%%%%%%%%%%%%%%%%


\newcommand{\eps}{\varepsilon}

\renewcommand{\c}[1]{\ensuremath{\EuScript{#1}}}
\renewcommand{\b}[1]{\ensuremath{\mathbb{#1}}}
\newcommand{\s}[1]{\textsf{#1}}

\newcommand{\E}{\textbf{\textsf{E}}}
\renewcommand{\Pr}{\textbf{\textsf{Pr}}}

\newenvironment{pkl}{%
\begin{itemize}%
%\vspace{-\topsep}%
\setlength\itemsep{-0.5\parskip}%
\setlength\parsep{0in}%
}{%
%\vspace{-\topsep}%
\end{itemize}}

\newenvironment{enu}{%
\begin{enumerate}%
\vspace{-\topsep}%
\setlength\itemsep{-\parskip}%
\setlength\parsep{0in}%
}{%
\vspace{-\topsep}%
\end{enumerate}}
\allowdisplaybreaks


\title{Readme for reproducibility submission of SIGMOD'21 paper ID 68}
\author{}
\date{}

\begin{document}
\maketitle

\section{Source code info}

{\bf Repository:} https://github.com/Yanqing-UTAH/ATTPCode\\
{\bf Contact: } Zhuoyue (zzhao35@buffalo.edu) \\
{\bf Programming Langauge:} mainly C/C++, some of the scripts are written in
Python\\
{\bf Software/library:}
\begin{pkl}
    \item (Required for our code) gcc/g++ >= 8 (requires support for
-std=gnu++17), make, lapack and lapacke, blas and cblas, fftw3
    \item (Required for Vertica baseline in Figure 1) unixodbc, Vertica
    \item (Required for creating datasets) python3, numpy, scipy,
sklearn and the common shell programs (bash, grep, sed, and etc.).
    \item (Required for plotting figures): jupyter notebook, python3, matplotlib, pandas, numpy
    \item (Optional, only if you need to regenerate the configure
        script): m4, autoconf, autoheader
\end{pkl}
{\bf Hardware requirement:} we assume the architecture implements <=
48-bit virtual address space and always uses x86-64 canonical
addresses.  That's the case for any combination of Intel/AMD processor
and Linux kernel, except for the kernels that are configured to allocate
memory beyond 47-bit user address space on the processors that
support 5-level paging (e.g., Intel Ice Lake).

\section{Test environment}
{\bf Software environment:}
\begin{pkl}
	\item Ubuntu 18.04.6 LTS
    \item GNU make 4.1
	\item GCC/G++ 8.3.0
	\item liblapack-dev 3.7.1
	\item liblapacke-dev 3.7.1
	\item libblas-dev 3.7.1
	\item libatlas-base-dev 3.10.3
	\item libfftw3-dev 3.3.7
    \item python3 3.6.9
    \item sklearn 0.22.2
    \item scipy 1.4.1
    \item numpy 1.19.0
    \item unixodbc-dev 2.3.4
    \item Vertica 10.0.1 Community Edition
\end{pkl}
Note that these are the ones installed at the time we performed the
experiments but it should be ok if you use newer versions. For Vertica
10.0.1 Community Edition, you may find the free community version at
\url{https://www.vertica.com/}.

{\bf Hardware environment:}


For all experiments in Section 6: we have 10 type-1 nodes and 6 type-2
nodes and we launched any of the experiments on any one of them
depending on the availability. They have comparable performance and we
made sure that there were no computation or I/O heavy programs running
concurrently.

\vspace{1mm}
\begin{tabular}{|l|l|l|}
	\hline
	Node type &  1 & 2 \\\hline
	CPU & Intel Core i7-3820 & Intel Xeon E5-1650 v3 \\\hline
	CPU Frequency & 3.6 GHz & 3.50 GHz \\\hline
	L1 Cache & 32KB + 32 KB & 32KB + 32 KB\\\hline
	L2 Cache & 256 KB & 256 KB\\\hline
	L3 Cache & 10 MB (shared) & 15 MB (shared) \\\hline
	Memory & 64GB (DDR3-1600 x 8) & 128GB (DDR4-2133 x 4) \\\hline
	Secondary storage & WD HDD 7200RPM 2TB & Seagate HDD 7200RPM 1TB
	\\\hline
	Network & not used & not used \\\hline
\end{tabular}
\vspace{1mm}

For Figure 1 scalability test against Vertica, we used the following
server to make it run faster. Note that this figure is only used for
introduction and is not used for comparison against any of the
existing persistent sketches.

\vspace{1mm}
\begin{tabular}{|l|l|}
    \hline
	Node type &  3 \\\hline
	CPU & \\\hline
	CPU Frequency & 3.6 GHz \\\hline
	L1 Cache & 32KB + 32 KB \\\hline
	L2 Cache & 256 KB \\\hline
	L3 Cache & 10 MB (shared)  \\\hline
	Memory & 64GB (DDR3-1600 x 8)  \\\hline
	Secondary storage & WD HDD 7200RPM 2TB \\\hline
	Network & not used \\\hline
\end{tabular}
\vspace{1mm}


\section{Preparing datasets}

We have two scripts for creating the datasets: 1) the datasets based
on the FIFA World Cup 98 website access logs for the ATTP/BITP heavy
hitter experiments; 2) the synthetic datasets for the ATTP
frequent direction experiments. 

For the world-cup datasets (for the ATTP/BITP heavy hitters), run:
\begin{verbatim}
$ ./data_proc/world-cup/prepare_data.sh
\end{verbatim}
It takes about 4 hours to download and generate all the datasets. In
case the website hosting the raw logs is unreachable, please contact
Zhuoyue (zzhao35@buffalo.edu) for our own copy.

For the matrix datasets (for the ATTP frequent directions): run:
\begin{verbatim}
$ ./data_proc/gen_mat_data.sh
\end{verbatim}
It takes less than 12 minutes to generate all the three datasets (small,
medium and large). If you only want one or some of the three datasets,
specify its name as a command line argument to the script (e.g.,
$\texttt{./data\_proc/gen\_mat\_data.sh small}$)

All the generated data are put into the data/ directory, see
data/README.md for descriptions.

\section{Prepare the system and building the code}
If you're using Ubuntu, you should be able to use the following to
prepare all the required prerequisites except Vertica:
\begin{verbatim}
$ sudo apt install gcc g++ make liblapacke-dev libatlas-base-dev \
> libfftw3-dev python3 python3-pip unixodbc unixodbc-dev
$ pip3 install sklearn scipy numpy
\end{verbatim}

To build our code:
\begin{verbatim}
$ ./configure
$ make
\end{verbatim}

To setup a single-node Vertica for the baseline in Figure 1, first
request and download the free vertica community version from
\url{https://www.vertica.com/} and install it in your system. In case
it's no longer available, we still have a copy on our server. We
provided a script $\texttt{vertica/setup\_vertica.sh}$ to help you set
it up on Ubuntu. Please take a look at the first a few lines of the
script (change them if needed) and do the following:
\begin{verbatim}
$ dpkg -i <path-to-vertica-deb-package>
$ sudo ./vertica/setup_vertica.sh # you might have to tweak it, see comments
$ sudo su verticadba
$ admintools # follow prompts to finish setup and create a DB
$ exit # back to your own account
$ /opt/vertica/bin/vsql <your-db-name> verticadba # use this to check if the DB is running
$ export VERTICAINI=${HOME}/.vertica.ini
\end{verbatim}
You'll also need to create two files $\mathtt{\sim/.odbc.ini}$ and
$\mathtt{\sim/.vertica.ini}$ for Vertica ODBC connection. We provided
a sample for them in $\texttt{vertica/}$ and you might want to tweak
them according to your own system setup.


\section{Running the experiments}

\section{Plotting the figures}

You'll need jupyter notebook, matplotlib, numpy and pandas to plot the
figures. The scripts provided in plot/ work on my local WSL 2.0
installation with Ubuntu 20.04 LTS, python 3.8.10, matplotlib 3.4.3,
numpy 1.21.3, pandas 1.3.4 and jupter notebook 6.0.3.
\begin{verbatim}
$ sudo apt install python3 python3-pip jupyter-notebook
$ pip3 install matplotlib numpy pandas
\end{verbatim}

The following table lists which notebook you should run to generate
specific figures as well as the expected input file. These scripts
also generates pdf files for the figures in the plot/ directory.

{\scriptsize
\begin{tabular}{|l|l|l|}
    \hline
    Notebook & Figures & Input \\\hline
    $\texttt{HH\_ATTP\_clientid.ipynb}$ & 2, 3(left), 4 &
    $\texttt{filtered\_logs/client\_id\_attp\_filtered\_combined.txt}$\\\hline
    $\texttt{HH\_ATTP\_objectid.ipynb}$ & 3(right), 5, 6 &
    $\texttt{filtered\_logs/object\_id\_attp\_filtered\_combined.txt}$\\\hline
    $\texttt{HH\_BITP\_clientid.ipynb}$ & 7, 8(left), 9 &
    $\texttt{filtered\_logs/client\_id\_bitp\_filtered\_combined.txt}$\\\hline
    $\texttt{HH\_BITP\_objectid.ipynb}$ & 8(right), 10, 11&
    $\texttt{filtered\_logs/object\_id\_bitp\_new\_filtered\_combined.txt}$\\\hline
    $\texttt{MAT\_ATTP\_small.ipynb}$ & 12(a), 13(left), 14&
    $\texttt{filtered\_logs/ms\_small\_attp\_filtered\_combined.txt}$ \\\hline
    $\texttt{MAT\_ATTP\_medium.ipynb}$ & 12(b), 13(right), 15&
    $\texttt{filtered\_logs/ms\_medium\_attp\_filtered\_combined.txt}$ \\\hline
    $\texttt{MAT\_ATTP\_big.ipynb}$ & 12(c), 16&
    $\texttt{filtered\_logs/ms\_big\_attp\_filtered\_combined.txt}$ \\\hline
    $\texttt{scalability\_test\_client\_id\_ATTP.ipynb}$ & 1 &
    $\texttt{scalability\_logs/scalability-test-client-id.log}$ \\\hline
\end{tabular}
}


\end{document}

